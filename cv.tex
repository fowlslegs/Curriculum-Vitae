% !TEX root
% !TEX program = xelatex
%
%%%%%%%%%%%%%%%%%%%%%%%%%%%%%%%%%%%%%%%%%
% Friggeri Resume/CV
% XeLaTeX Template
% Version 1.1 (9/2/15)
%
% This template has been downloaded from:
% http://www.LaTeXTemplates.com
%
% Original author:
% Adrien Friggeri (adrien@friggeri.net)
% https://github.com/afriggeri/CV
%
% License:
% CC BY-NC-SA 3.0 (http://creativecommons.org/licenses/by-nc-sa/3.0/)
%
% Important notes:
% This template needs to be compiled with XeLaTeX and the bibliography, if used,
% needs to be compiled with biber rather than bibtex.
%
%%%%%%%%%%%%%%%%%%%%%%%%%%%%%%%%%%%%%%%%%

% Options
% 'print': remove colors from this template for printing
% 'nocolor' to disable colors in section headers
\documentclass[]{fowlslegs-cv}

\usepackage{graphicx}
\newcommand\makeIcon[1]{%
    \raisebox{-1pt}{%
        \includegraphics[width=0.83\baselineskip, keepaspectratio]{icons/#1}%
    }%
}
\newcommand\makeIconB[1]{%
    \raisebox{-1pt}{%
        \includegraphics[height=0.83\baselineskip, keepaspectratio]{icons/#1}%
    }%
}

\usepackage{enumitem}
\setlist[itemize]{nosep, topsep=0pt,leftmargin=*}

\begin{document}
\header{Noah}{Vesely}{Cryptographer} % Your name and current job title/field
%----------------------------------------------------------------------------------------
%	SIDEBAR SECTION
%----------------------------------------------------------------------------------------
\begin{aside} % In the aside, each new line forces a line break
\section{Contact}
\makeIcon{address} \hfill London, UK
\makeIcon{phone} \hfill \href{tel:+44 7903 224172}{+44 7903 224172}\vspace{1pt}
\makeIcon{email} \hfill \href{mailto:fowlslegs@riseup.net}{fowlslegs@riseup.net}\vspace{1pt}
\makeIcon{github} \hfill \href{https://github.com/nvesely}{github.com/nvesely}\vspace{1pt}
%------------------------------------------------
\section{Languages}
Rust, Go, Python, C/C++, SQL, Clojure, Shell, LaTeX
%------------------------------------------------
\section{Technologies}
\textbf{Cryptography.} libzexe, bellman, libsodium
\textbf{Security.} Tor, IPTables, OSSEC, tcpdump/ Wireshark
\textbf{Data Science.} scikit-learn, Jupyter Notebooks, NumPy, Clojush
\textbf{Devops.} Docker, KVM, Container Linux, libvirt, Ansible, Vagrant, systemd, VirtualBox
\textbf{Frameworks.} Django, Flask
\textbf{Revision.} git, Github
\textbf{Development.} Agile, TDD
\textbf{Firmware.} Coreboot
\section{Interests}
Zero-knowledge proofs
Anonymous communication
Machine learning
Functional programming
Formal verification
Guitar
Rock climbing
\end{aside}
%----------------------------------------------------------------------------------------
%	WORK EXPERIENCE SECTION
%----------------------------------------------------------------------------------------
\section{Experience}
\begin{entrylist}
%------------------------------------------------
\entry
{Sum. '19}
{Celo}
{Research Scientist}
{San Francisco, USA}
{Research and design of an ultra-light client for a BFT-style blockchain network
utilizing zkSNARKS, proof-carrying data, and SNARK-friendly primitives.}
%------------------------------------------------
\entry
{Sum. '18}
{Glyff}
{Research Scientist}
{London, UK}
{Designed a privacy-preserving smart contract platform based on Ethereum.
Co-authored a whitepaper. Audited a prototype implementation.}
%------------------------------------------------
\entry
{2017--8}
{Freelance}
{Security and Cryptography Engineer}
{Mexico City, MX}
{Information security and cryptography related development, consulting, and
training. Clients included Data Cívica and Human Rights Data Analysis Group.}
%------------------------------------------------
\entry
{2015--7}
{Freedom of the Press Foundation}
{Security Engineer}
{San Francisco, USA}
{\makeIcon{github}
  \href{https://github.com/freedomofpress/securedrop}{/securedrop}. Developed
  an open-source whistleblower submission platform. Stringent security
  requirements and a multi-machine, multi-OS architecture demanded
  wide-breadth domain knowledge including cryptographic, network, OS, and
  application-level security expertise. \\
%------------------------------------------------
  \makeIcon{github}
  \href{https://github.com/freedomofpress/fingerprint-securedrop}{/fingerprint-securedrop}.
  Designed and implemented a machine learning (ML) system to evaluate website
  fingerprinting attacks and defenses for Tor onion services. Led Tor meeting
  group sessions on the topic and presented my work to KU Leuven's COSIC
  research group. \\
%------------------------------------------------
  \makeIcon{github}
  \href{https://github.com/freedomofpress/wa-knn-fingerprint-securedrop}{/wa-knn-fingerprint-securedrop}.
  The fastest implementation of a specialized ML classifier.}
%------------------------------------------------
\entry
{2013--}
{Web Precision}
{Security Consultant \& Systems Administrator}
{Mission Viejo, CA, USA}
{Intermittent consulting. Most recently (2018): designed a sandboxing solution for a multi-tenant web server that succesfully isolated website compromises to unprivileged tenant domains.}
%------------------------------------------------
\entry
{2012--5}
{Hampshire College Quantitative Resource Center}
{Manager}
{Amherst, MA, USA}
{Tutored primarily mathematics and computer science. Managed the hiring process,
budgeting, and staff schedule. Under my oversight the center saw record
attendance levels.}
%------------------------------------------------
\end{entrylist}
%----------------------------------------------------------------------------------------
%	PROJECTS SECTION
%----------------------------------------------------------------------------------------
\section{Papers}
\begin{entrylist}
%------------------------------------------------
\entryy
{2019}
{\textsc{Usha}: Preprocessing zkSNARKs with Universal and Updatable SRS}
{A. Chiesa, Y. Hu, M. Maller, P. Mishra, N. Vesely, N. Ward}
{Submitted to OAKLAND~19}
{We present a new methodology to construct preprocessing zkSNARKs where the
structured reference string (SRS) is universal and updatable, and use it to obtain a preprocessing zkSNARK where the SRS has linear size and arguments have constant size. Our construction improves on Sonic [Maller et al., CCS~2019], the prior state of the art in this setting, in all efficiency parameters.}
%------------------------------------------------
\entryy
{2019}
{Logarithmic proofs for Pairing Based Languages}
{Author list not yet finalized}
{In progress}
{We introduce a new transparent argument for pairing-based languages.}
\end{entrylist}
%------------------------------------------------
%------------------------------------------------
%------------------------------------------------
\section{Projects}
\begin{entrylist}
%------------------------------------------------
\entryy
{2018}
{Winternitz}
{Author}
{\makeIcon{github}\href{https://github.com/nvesely/winternitz}{/winternitz}}
{The first standalone implementation of the post-quantum WOTS-T one-time signature scheme.}
%------------------------------------------------
\entryy
{2018}
{Shamir's Secret Sharing}
{Author}
{\makeIcon{github}\href{https://github.com/SpinResearch/RustySecrets}{/RustySecrets}}
{An implementation of Shamir's Secret Sharing Scheme that provides authentication of shares.}
%------------------------------------------------
\entryy
{2018}
{SodiumOxide}
{Maintainer \& Contributor}
{\makeIcon{github}\href{https://github.com/sodiumoxide/sodiumoxide}{/sodiumoxide}}
{Rust bindings to the C++ libsodium cryptography library.}
%------------------------------------------------
\entryy
{2017--}
{Application Layer Padding Concerns Adversaries (ALPaCA)}
{Author}
{\makeIcon{github}\href{https://github.com/camelids/libalpaca}{/libalpaca}}
{A library that implements ALPaCa, an application-layer defense against website fingerprinting.}
%------------------------------------------------
\end{entrylist}
%----------------------------------------------------------------------------------------
%	EDUCATION SECTION
%----------------------------------------------------------------------------------------
\pagebreak
\section{Education}
\begin{entrylist}
\entry
{2019}
{\makeIcon{diploma} Master of Science {\normalfont\footnotesize in Information Security}}
{}
{University College London}
{\emph{On the Design of Polynomial Commitment Schemes}\\
  We introduce two new extractable polynomial commitments (PCs), one \emph{succinct} and with \emph{updatable} SRS, and another \emph{transparent} and practically efficient with logarithmic proofs and commitments, and sublinear verification. We present two new black-box transformations that given an extractable (succinct) PC scheme supporting evaluation of committed polynomials at a single point, produces a (succinct) scheme supporting evaluation of committed polynomials at a \DoQuote{query set} of points while guaranteeing \emph{per-polynomial degree-bounded extractability}. We formalize these security and efficiency properties in new definitions. \vspace{1mm}\\

%------------------------------------------------
  \emph{Other highlights:}
  \begin{itemize}
    \item Started two research papers in zero-knowledge proofs, one of which has been submitted to OAKLAND 2020 and the other still in progress.
    \item Currently implementing a novel zero-knowledge proof system for R1CS introduced in one of our research papers.
  \end{itemize}
}
%------------------------------------------------
\entry
{2015}
{\makeIcon{diploma} Bachelor of Science {\normalfont\footnotesize in Mathematics, minor in Computer Science}}
{}
{Hampshire College}
{\emph{McEliece-Type Cryptosystems as Post-Quantum Standards}\\
  An analysis of McEliece-Type cryptosystems as a post-quantum replacement for
  RSA. Review of literature focusing on developments towards an IND-CCA2 variant
  with sufficiently small key size for embedded devices. \vspace{1mm}\\
%------------------------------------------------
  \emph{An Evolved Cryptographic Compression Function}\\
  Used the PushGP genetic programming environment to evolve a cryptographic
  compression function. \vspace{1mm}\\
%------------------------------------------------
  \emph{Other highlights:}
  \begin{itemize}
    \item Worked in a genetic programming research group.
    \item Founded a student group to promote open-source \& privacy enhancing technologies.
\end{itemize}}
\end{entrylist}
%----------------------------------------------------------------------------------------
%----------------------------------------------------------------------------------------
\end{document}
